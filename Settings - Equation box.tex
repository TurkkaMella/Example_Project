% This line is already defined in "Settings - Theorem"
% \newtheorem{theorem}{Theorem}[section]

% Set up default settings for mdframed
\mdfsetup{
  linewidth=1pt,            % Width of the frame line
  roundcorner=10pt,         % Amount of rounding of the corners
  leftmargin=0,             % Space before the left margin
  rightmargin=0,            % Space before the right margin
  backgroundcolor=white!20, % Background color of the frame
  linecolor=black           % Color of the frame line
}
% Example to box something using the default settings
%   \begin{mdframed}
%   This text is framed with global settings.
%   \end{mdframed}

% Define style format1
\mdfdefinestyle{equation_style_box}{  % "equation_style" can be named anything you like
    linecolor=blue,
    linewidth=2pt,
    leftmargin=10pt,
    rightmargin=10pt,
    backgroundcolor=gray!5,
    roundcorner=10pt
}
% Define style format2
\mdfdefinestyle{theorem_style_box}{
    linecolor=red,
    linewidth=3pt,
    leftmargin=15pt,
    rightmargin=15pt,
    backgroundcolor=yellow!10,
    roundcorner=15pt
}
% Define style format3
\mdfdefinestyle{text_style_box}{
    linecolor=green,
    linewidth=1pt,
    leftmargin=5pt,
    rightmargin=5pt,
    backgroundcolor=cyan!5,
    roundcorner=5pt
}
% These defined styles can be used like this:

% Using format1
%   \begin{mdframed}[style=equation_style_box]
%   This content is framed with style format1.
%   \end{mdframed}

% Using format2
%   \begin{mdframed}[style=theorem_style_box]
%   This content is framed with style format2.
%   \end{mdframed}
  
% Using format3
%   \begin{mdframed}[style=text_style_box]
%   This content is framed with style format3.
%   \end{mdframed}

% If you would want, you could add a command like this
%   \newmdtheoremenv[style=text_style_box]{textBox}{Theorem}
% After which, if you want to use the format3, you would use this syntax
%   \begin{textBox}
%   This is a framed text with style format3.
%   \end{textBox}