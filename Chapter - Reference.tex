\section{Referencing Examples}

One of the fundamental aspects of LaTeX is its ability to handle references dynamically. This feature is particularly useful when dealing with mathematical equations, figures, tables, and even sections of text, such as appendices. A prime example of this is Equation~\ref{eq:gravity}, which introduces the concept of gravitational force. Similarly, we can reference sections of our document, such as the appendix on Nikola Tesla, which can be found in Appendix~\ref{app:Tesla}. The ability to reference dynamically ensures that our text always points to the correct equation or section, regardless of any modifications that might occur in the document structure.

Moreover, LaTeX's referencing capabilities extend to documenting historical academic contributions. For instance, Albert Einstein's seminal work on the theory of relativity \cite{einstein_relativity} revolutionized our understanding of space, time, and gravity. Meanwhile, Nikola Tesla's innovative experiments paved the way for the modern use of alternating current \cite{tesla_transmission}, and his life continues to be a subject of great interest as detailed in various biographies \cite{tesla_biography}. 

This referencing system not only streamlines the writing process but also enhances the document's accuracy and readability. It exemplifies LaTeX's power in managing complex documents, making it an indispensable tool for academic writing and technical documentation.

