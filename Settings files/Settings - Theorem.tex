% Note that these definitions are totally arbitrary, meaning that you could 
% define anything like this. For example you could have:
%   \theoremstyle{myAss}
%   \newtheorem{myAss}[theorem]{myAss}
% and you could use this with the syntax
%   \begin{myAss}
%   Let \( f: A \to B \) be a function from set \( A \) to set \( B \). If \( f \) is bijective, then there exists an inverse function \( f^{-1}: B \to A \).
%   \end{myAss}

% This defines 
%   - "theorem" to be the keyword in commands like: \begin{theorem} ... \end{theorem}
%   - "Theorem" to be the text that appears in the PDF with the numbering
%   - "section" to be the father category of theorem, so that children of theorem will be in the same 
%               numbering, meaning that if Lemma would be defined with syntax \newtheorem{lemma}{Lemma}[section]
%               it would have its own numbering, since now theorem is not its father but brother, hence they 
%               have now separate numberings.  
\newtheorem{theorem}{Theorem}[section]
\newtheorem{lemma}[theorem]{Lemma}
\newtheorem{corollary}[theorem]{Corollary}
\newtheorem{proposition}[theorem]{Proposition}

% The \theoremstyle{definition} command was used before defining the definition environment to 
% ensure that definitions are displayed differently (in upright text) than theorems, lemmas, 
% and corollaries (which are typically in italic text). The lemma did not need a \theoremstyle 
% command before it because lemmas, being similar to theorems in nature, use the default plain style, 
% which is already applied.
\theoremstyle{definition}
\newtheorem{definition}[theorem]{Definition}

\theoremstyle{remark}
\newtheorem{remark}[theorem]{Remark}

\theoremstyle{plain}
\newtheorem{plain}[theorem]{Plain}

% You can only use definition, remark, plain with this syntax since they are default. 