\section{Mathematical Proofs and Theorems}

Mathematical proofs and theorems are the core components of any mathematical text. They provide the logical foundation upon which mathematical theory is built. LaTeX, with the help of packages like `amsthm`, provides an excellent way to typeset these structures.

\subsection{Sample Theorem}
% To have this boxed  
\begin{mdframed}
\begin{theorem}
    If \( f: A \to B \) is bijective, then there exists an inverse function \( f^{-1}: B \to A \).
\end{theorem}
\end{mdframed}

\begin{theorem}
Let \( f: A \to B \) be a function from set \( A \) to set \( B \). If \( f \) is bijective, then there exists an inverse function \( f^{-1}: B \to A \).
\end{theorem}

\begin{proof}
Since \( f \) is bijective, for each \( b \in B \), there exists a unique \( a \in A \) such that \( f(a) = b \). We define \( f^{-1}(b) = a \). To show that \( f^{-1} \) is the inverse of \( f \), we must demonstrate that \( f(f^{-1}(b)) = b \) for all \( b \in B \) and \( f^{-1}(f(a)) = a \) for all \( a \in A \).

First, we show that \( f(f^{-1}(b)) = b \) for all \( b \in B \). Since \( f^{-1}(b) = a \), it follows that \( f(f^{-1}(b)) = f(a) = b \).

Next, we show that \( f^{-1}(f(a)) = a \) for all \( a \in A \). Since \( f(a) = b \), and \( f^{-1}(b) = a \), it follows that \( f^{-1}(f(a)) = f^{-1}(b) = a \).

Therefore, \( f^{-1} \) is the inverse function of \( f \).
\end{proof}

\subsection{Sample Lemma}
\begin{lemma}
For every non-empty set \( S \), if \( f: S \to S \) is injective, then \( f \) is surjective.
\end{lemma}

\begin{proof}
(Sketch) Since \( S \) is non-empty and \( f \) is injective, every element of \( S \) maps to a distinct element of \( S \). Given the finiteness of \( S \), this implies that \( f \) must cover all elements of \( S \), hence it is surjective.
\end{proof}


\subsection{Sample Corollary}
\begin{corollary}
If \( f: A \to B \) and \( g: B \to C \) are bijective functions, then the composition \( g \circ f: A \to C \) is also bijective.
\end{corollary}

\begin{proof}
Omitted for brevity.
\end{proof}

\subsection{Sample Definition}
\begin{definition}
A function \( f: A \to B \) is said to be \emph{injective} if for every \( a_1, a_2 \in A \), \( f(a_1) = f(a_2) \) implies that \( a_1 = a_2 \).
\end{definition}

\subsection{Sample Remark}
\begin{remark}
This is a remark that provides additional information or context about the material presented above.
\end{remark}
